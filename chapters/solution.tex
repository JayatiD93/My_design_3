\newcommand{\Cross}{\mathbin{\tikz [x=1.4ex,y=1.4ex,line width=.2ex] \draw (0,0) -- (1,1) (0,1) -- (1,0);}}%
%\renewcommand{\theequation}{\theenumi}
%\begin{enumerate}[label=\thesection.\arabic*.,ref=\thesection.\theenumi]
%\numberwithin{equation}{enumi}
%	
%	
%
%\item $\triangle AMC \cong \triangle DMB$  by SAS congruency $\because$
%\begin{enumerate}
%\item $AM = BM$
%\item $CM = DM$
%\item $\phase{AMC}$ = $\phase{DMB}$ ( Vertically Opposite Angles)
%\end{enumerate}
%%
%\item From \eqref{eq:constr_b}, \eqref{eq:constr_c} and \eqref{eq:constr_d},
%%
%%
%\begin{align}
%\brak{\vec{D}-\vec{B}}^T
%\brak{\vec{B}-\vec{C}} &= \myvec{0 & b}\myvec{a \\ 0} = 0
%\\
%\implies BD \perp BC
%\end{align}
%%
%\item From \eqref{eq:constr_a}, \eqref{eq:constr_b}, \eqref{eq:constr_c} and \eqref{eq:constr_d},
%\begin{align}
%\norm{\vec{A}-\vec{B}} &= \norm{\myvec{-a \\ b}}
%\\
%\norm{\vec{C}-\vec{D}} &= \norm{\myvec{-a \\ -b}}
%\\
%\implies \norm{\vec{A}-\vec{B}} &= \norm{\vec{C}-\vec{D}}\\
%\text{or, } AB &=CD
%\label{eq:solution_abcd}
%\end{align}
%%
%From RHS congruence,  $\triangle ACB \cong  \triangle DCB$.
%\item From \eqref{eq:solution_abcd}, noting that $\vec{M}$ is the mid point of both $AB$ and $CD$, 
%\begin{align}
%CM = \frac{1}{2}CD =\frac{1}{2} AB
%\end{align}
%
%
%
%\end{enumerate}


%
%The area of a parallelogram can be defined as:\\
%$Area$ = $\abs{\vec{a} \Cross \vec{b}} $
%
%%\begin{equation}
%%\begin{aligned}
%%    \vec{a} \Cross \vec{b} = \begin{vmatrix}
%%      $\^i$       & $\^j$    & $\^k$ \\ 
%%      3       & 1    & 4 \\
%%      1       & -1     & 1 \\
%%      
%%    \end{vmatrix}
%%\end{aligned}
%%\end{equation}
%%So, $\vec{a} \Cross \vec{b}$ = 5\^i + \^j - 4\^k.\\
%%
%%
%%Now, $\abs{\vec{a} \Cross \vec{b}} $ = $\sqrt{5^2 + 1^2 + (-1)^2}$\\
%%\text{or, } $\abs{\vec{a} \Cross \vec{b}} $ = $\sqrt{27}$\\
%%\text{or, } $Area$ = $\sqrt{27}$\\
%%Hence, the area of the parallelogram in the above problem statement is $\sqrt{27}$.\\
%%\text{or, }\\
%The cross-product can be calculated as:\\
%$\vec{a} \Cross \vec{b} $ = $[\vec{a}]_x \vec{b}$
%where $[\vec{a}]_x$ = $\vec{a} \Cross $ \^e 
%and \^e is the unit vector. \\
%If $\vec{a}$ can be expressed as:\\
%\begin{equation}
%\begin{aligned}
%      \vec{a} = \begin{pmatrix}
%      a_1 \\ 
%      a_2 \\
%      a_3 \\
%      
%    \end{pmatrix} = \begin{pmatrix}
%      3 \\ 
%      1 \\
%      4 \\
%      
%    \end{pmatrix}
%\end{aligned}
%\end{equation} and 
%\begin{equation}
%\begin{aligned}
%      \vec{b} = \begin{pmatrix}
%      b_1 \\ 
%      b_2 \\
%      b_3 \\
%      
%    \end{pmatrix} = \begin{pmatrix}
%      1 \\ 
%      -1 \\
%      1 \\
%      
%    \end{pmatrix}
%\end{aligned}
%\end{equation}
%Then $[\vec{a}]_x$ can be expressed as:\\
%%$[\vec{a}]_x$ = $ [\vec{a} \Cross $  \^i $\quad$  $\vec{a} \Cross $ \^j $\quad$ $\vec{a} \Cross $ \^k] 
%\begin{equation}
%\begin{aligned}
%       \text{or, }[\vec{a}]_x = \begin{pmatrix}
%      0       & -a_3    & a_2 \\ 
%      a_3       & 0    & -a_1 \\
%      -a_2       & a_1     & 0 \\
%      
%    \end{pmatrix}= \begin{pmatrix}
%      0       & -4    & 1 \\ 
%      4       & 0    & -3 \\
%      -1       & 3     & 0 \\
%      
%    \end{pmatrix}
%\end{aligned}
%\end{equation}
%So, the $[\vec{a}]_x \vec{b}$ can be calculated as:\\
%\begin{equation}
%\begin{aligned}
%      [\vec{a}]_x \vec{b} = \begin{pmatrix}
%      0       & -4    & 1 \\ 
%      4       & 0    & -3 \\
%      -1       & 3     & 0 \\
%      
%    \end{pmatrix} \begin{pmatrix}
%      1\\ 
%      -1\\
%      1\\
%      
%    \end{pmatrix} = \begin{pmatrix}
%      5\\ 
%      1\\
%      -4\\
%      
%    \end{pmatrix} 
%\end{aligned}
%\end{equation}
%%\text{or, } $[\vec{a}]_x \vec{b}$ = 5\^i + \^j - 4\^k.\\
%Now, $\abs{[\vec{a}]_x \vec{b}} $ = $\sqrt{5^2 + 1^2 + (-1)^2}$\\
%\text{or, } $\abs{[\vec{a}]_x \vec{b}} $ = $\sqrt{27}$\\
%\text{or, } $Area$= $\abs{\vec{a} \Cross \vec{b}} $ = $\abs{[\vec{a}]_x \vec{b}} $ = $\sqrt{27}$\\
%Hence, the area of the parallelogram in the above problem statement is $\sqrt{27}$.\\

From the problem statement, we got that:\\

\begin{equation}
\begin{aligned}
2a+b=4\\
\end{aligned}
\end{equation}

\begin{equation}
\begin{aligned}
a-2b=-3\\
\end{aligned}
\end{equation}

\begin{equation}
\begin{aligned}
2c-d=11\\
\end{aligned}
\end{equation}

\begin{equation}
\begin{aligned}
4c+3d=24\\
\end{aligned}
\end{equation}

These equations can be written as:\\

\begin{equation}
\begin{aligned}
      \begin{pmatrix}
      2       & 1    & 0   & 0\\ 
      1       & -2   & 0   & 0\\
      0       & 0    & 5   & -1\\ 
      0       & 0   & 4   & 3\\
            
    \end{pmatrix} \begin{pmatrix}
      a\\ 
      b\\
      c\\
      d\\
            
    \end{pmatrix} = \begin{pmatrix}
      4\\ 
      -3\\
      11\\
      24\\
            
    \end{pmatrix} 
\end{aligned}
\end{equation}

So the coefficient matrix $A$ can be expressed as:\\

\begin{equation}
\begin{aligned}
    A=  \begin{pmatrix}
      2       & 1    & 0   & 0\\ 
      1       & -2   & 0   & 0\\
      0       & 0    & 5   & -1\\ 
      0       & 0   & 4   & 3\\
            
    \end{pmatrix} 
\end{aligned}
\end{equation}
 
And the augmented matrix $B$ can be expressed as:\\

\begin{equation}
\begin{aligned}
    B=  \begin{pmatrix}
      2       & 1    & 0   & 0  & 4\\ 
      1       & -2   & 0   & 0  & -3\\
      0       & 0    & 5   & -1  & 11\\ 
      0       & 0   & 4   & 3   & 24\\
            
    \end{pmatrix}
\end{aligned}
\end{equation}

Now, if we express the augmented matrix as Echelon form, then it will be:\\

\begin{equation}
\begin{aligned}
    B=  \begin{pmatrix}
      2       & 1    & 0   & 0  & 4\\ 
      1       & -2   & 0   & 0  & -3\\
      0       & 0    & 5   & -1  & 11\\ 
      0       & 0   & 4   & 3   & 24\\
            
    \end{pmatrix}\\
    \xleftrightarrow
    [R_4\leftarrow R_4 - R_3]{R_2\leftarrow 2R_2 - R_1} 
    \begin{pmatrix}
      2       & 1    & 0   & 0  & 4\\ 
      0       & -5   & 0   & 0  & -10\\
      0       & 0    & 5   & -1  & 11\\ 
      0       & 0   & -1   & 4   & 13\\
            
    \end{pmatrix}\\
    \xleftrightarrow
    [R_4\leftarrow 5R_4 + R_3]{R_2\leftarrow \frac{R_2}{(-5)}}\\
    \begin{pmatrix}
      2       & 1    & 0   & 0  & 4\\ 
      0       & 1   & 0   & 0  & 2\\
      0       & 0    & 5   & -1  & 11\\ 
      0       & 0   & 0   & 19   & 76\\
            
    \end{pmatrix}\\
    \xleftrightarrow
    []{R_4\leftarrow \frac{R_4}{(19)}}\\
    \begin{pmatrix}
      2       & 1    & 0   & 0  & 4\\ 
      0       & 1   & 0   & 0  & 2\\
      0       & 0    & 5   & -1  & 11\\ 
      0       & 0   & 0   & 1   & 4\\
            
    \end{pmatrix}
\end{aligned}
\end{equation}

So, from here we can say that the the rank of $A$ = 4 and the rank of $B$ = 4.
So the equations have unique solution.

\begin{equation}
\begin{aligned}
      \begin{pmatrix}
      2       & 1    & 0   & 0\\ 
      0       & 1   & 0   & 0\\
      0       & 0    & 5   & -1\\ 
      0       & 0   & 0   & 1\\
            
    \end{pmatrix} \begin{pmatrix}
      a\\ 
      b\\
      c\\
      d\\
            
    \end{pmatrix} = \begin{pmatrix}
      4\\ 
      2\\
      11\\
      4\\
            
    \end{pmatrix} 
\end{aligned}
\end{equation}
This indicates that:\\
$2a+b$ = 4\\
and $b$ = 2\\
$\implies  a$ = 1.\\
And, \\
$5c-d$ = 11\\
and $d$ = 4\\
$\implies  c$ = 3.\\
Hence, a = 1, b= 2, c= 3 and d= 4.


